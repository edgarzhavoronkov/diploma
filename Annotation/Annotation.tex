\documentclass[12pt]{article}
\usepackage{fontspec}
\usepackage{setspace}
\usepackage{polyglossia}
\usepackage{titlesec}
\setdefaultlanguage{russian}
\setmainfont[Mapping=tex-text]{CMU Serif}
\author{Жаворонков Эдгар \\ группа 4501}
\title{Предварительная аннотация к выпускной квалификационной работе бакалавра}
\date{24 ноября 2014 г.}
\titlespacing\section{0pt}{12pt plus 4pt minus 2pt}{0pt plus 2pt minus 2pt}
\singlespacing
\begin{document}
    %\maketitle
    %\newpage
    %\tableofcontents
    %\newpage
    \section{Предварительная формулировка темы.}
        Разработка оптимизатора запросов для колонно-ориентированной in-memory базы данных.
    \section{Цель ВКР.}
        Целью выпускной квалификационной работы является разработка оптимизатора запросов для колонно-ориентированной in-memory базы данных
    \section{Краткое описание прикладного процесса, для автоматизации которого будет предназначен разрабатываемый компонент.}
        %Спросить, есть ли на кафедре большие спецы по VoltDB, Vertica
<<<<<<< HEAD
        %неплохо бы написать, почему стандартные решения не подходят
        Разрабатываемый компонент является частью базы данных общего назначения, специализированной для поиска по большому количеству критериев. Подобную базу можно использовать для большого количества разнообразных задач, в первую очередь для организации поиска по товарам в больших торговых площадках (wikimart.ru - 1.8 млн товаров, Яндекс.Маркет - 57 млн. предложений, amazon.com - 36 млн книг).

        Для подобных задач характерен поиск по большому количеству критериев(около ста для одного предложения Яндекс.Маркета, двадцати - для amazon.com) и разные шаблоны поиска по критериям (поиск по диапазону, по совпадению, полнотекстовый поиск). Задача осложняется неравномерным распределением критериев в данных и потенциально большими размером результирующих выборок. При этом, к базе данных предъявляются высокие требования к производительности(как и к нагрузке, так и к скорости получения результата). Стандартные реляционные базы данных, как и NoSQL не удовлетворяют таким требованиям.

        Система управления базами данных Vindur предназначена для решения задач быстрого многокритериального поиска и представляет собой хранящееся в памяти колонно-ориентированное хранилище. Колонно-ориентированное представление данных подразумевает расположение колонок таблицы в памяти друг за другом и предназначено для решения задач сложного поиска по нескольки критериям  (см. статью ``C-store: a column-oriented DBMS.'' Mike Stonebraker, Daniel J. Abadi, Adam Batkin, Xuedong Chen, Mitch Cherniack, Miguel Ferreira, Edmond Lau, Amerson Lin, Sam Madden, Elizabeth O'Neil, Pat O'Neil, Alex Rasin, Nga Tran, Stan Zdonik. August 2005. VLDB '05: Proceedings of the 31st international conference on Very large data bases)     
=======
        Разрабатываемый компонент является частью специализированной базы данных, предназначенной для эффективной
        реализации поиска по большому количеству критериев. Подобная база данных может использоваться для большого
        количества разнообразных задач, в первую очередь для организации поиска по товарам в больших торговых
        площадках (wikimart.ru - 1.8 млн товаров, Яндекс.Маркет - 57 млн. предложений, amazon.com - 36 млн книг). Для
        подобных задач характерен поиск по большому количеству критериев (около ста для одного предложения
        Яндекс.Маркета, двадцати - для amazon.com) и разнообразные принципы поиска по критериям (поиск по диапазону,
        по совпадению, полнотекстовый поиск). Задача осложняется неравномерным распределением критериев в данных
        и потенциально большими размером результирующих выборок. При этом, к базе данных предъявляются высокие
        требования к производительности (как к нагрузке, так и к скорости получения результата). Стандартные
        реляционные базы данных не удовлетворяют таким требованиям, современные NoSQL решения так же не реализуют
        необходимую функциональность.
>>>>>>> 657ddd360ff2d7aad74ee004619fc524e438d8a8
    \section{Требования к разрабатываемому компоненту.}
        Функциональные требования:
        \begin{enumerate}\itemsep1pt \parskip0pt \parsep0pt
            \item Генерация плана выполнения запроса
            \item Возможность модификации плана в ходе исполнения
            \item Возможность подстройки оптимизатора под конкретные запросы
            \item Возможность настройки сложности оптимизации
            \item Возможность подстройки оптимизатора под конкретное железо
        \end{enumerate}
        
        Формальные показатели: %ничего не понятно (1)
        \begin{enumerate}\itemsep1pt \parskip0pt \parsep0pt 
            \item Увеличение скорости выполнения запроса в два раза относительно отсутствия оптимизатора
        \end{enumerate}
        
        Нефункциональные требования: %ничего не понятно (2)
        \begin{enumerate}\itemsep1pt \parskip0pt \parsep0pt 
            \item Нетребовательность к ресурсам (не составлять план запроса дольше, чем исполнять его)
        \end{enumerate}
    \section{Руководитель. В каком объеме предоставлена информация для выполнения ВКР.}
        Научный руководитель и заказчик системы - Дельгядо Филипп Игоревич, куратор академических программ, JetBrains. Имеется возможность непосредственного взаимодействия, техническая документация Oracle Java и стандарты кодирования Java Coding Conventions
    \section{Описание технологий, которые будут использованы при разработке.}
        Язык реализации - Java. Некоторые фреймворки и библиотеки - Spring, JUnit, JProfiler, Apache Lucene, Java EWAH. Система контроля версий - Git. Репозиторий - github.com 
    \section{Описание методов тестированя и оценки качества разработанного компонента.}
        Тестирование - Performance, Unit testing. Так же, для оценки качества оптимизатора и базы в целом предполагается использование генератора тестовых данных и реальных данных, взятых из открытого источника.
    \section{Описание результатов работы.}
        \begin{enumerate}\itemsep1pt \parskip0pt \parsep0pt
            \item Артефакты проектирования:
                \begin{enumerate}\itemsep1pt \parskip0pt \parsep0pt 
                    \item Различные алгоритмы оптимизации
                    \item Диаграммы UML программной архитектуры
                \end{enumerate}
            \item Артефакты разработки:
                \begin{enumerate}\itemsep1pt \parskip0pt \parsep0pt 
                    \item Исходный код в репозитории
                \end{enumerate}
            \item Документация по системе
            \item Описание различных подходов к оптимизации.
            \item Сравнение различных алгоритмов оптимизации на тестовых данных
        \end{enumerate}
\end{document}
