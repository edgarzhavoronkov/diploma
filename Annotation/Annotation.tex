\documentclass[12pt]{article}
\usepackage{fontspec}
\usepackage{setspace}
\usepackage{polyglossia}
\usepackage{titlesec}
\setdefaultlanguage{russian}
\setmainfont[Mapping=tex-text]{CMU Serif}
\author{Жаворонков Эдгар \\ группа 4501}
\title{Предварительная аннотация к выпускной квалификационной работе бакалавра}
\date{24 ноября 2014 г.}
\titlespacing\section{0pt}{12pt plus 4pt minus 2pt}{0pt plus 2pt minus 2pt}
\singlespacing
\begin{document}
    %\maketitle
    %\newpage
    %\tableofcontents
    %\newpage
    \section{Предварительная формулировка темы.}
        Реализация эффективного выполнения поисковых запросов по множеству полей для колоночно-ориентированной базы данных, хранящиейся в памяти.
    \section{Цель ВКР.}
        Система управления базами данных Vindur представляет собой легко расширяемую колоночно-ориентированную базу данных, ориентированную на задачу быстрого многокритериального поиска. Архитектура базы данных Vindur предполагает хранение всех аттрибутов независимо, в отдельных хранилищах, адаптированных для быстрого выполнения конкретных поисковых операций (поиск по точному значению, поиск в диапазоне, полнотекстовый поиск и т.п.). Запросы к данным предполагают последовательный поиск подмножеств элементов в отдельных хранилищ с последующим пересечением множеств результатов. Подробности архитектуры колоночно-ориентированных баз данных см. в статье ``C-store: a column-oriented DBMS.'' Mike Stonebraker и др.
        Целью ВКР является написания модуля, организующего эффективное выполнение сложных запросов в базе данных Vindur. На эффективность выполнения запроса оказывает влияние множество факторов: распределение значений аттрибута в общем корпусе данных, параллельность, обеспечение транзакционной целостности, относительная сложность отдельных запросов к хранилищам, возможность кэширования промежуточных результатов и т.п.
    \section{Краткое описание прикладного процесса, для автоматизации которого будет предназначен разрабатываемый компонент.}
        %Спросить, есть ли на кафедре большие спецы по VoltDB, Vertica
        %неплохо бы написать, почему стандартные решения не подходят
        Разрабатываемый компонент является частью базы данных, предназначенной для эффективной реализации поиска по большому количеству критериев. Подобная база данных может использоваться для большого количества разнообразных задач, в первую очередь для организации поиска по товарам в больших торговых площадках (wikimart.ru - 1.8 млн товаров, Яндекс.Маркет - 57 млн. предложений, amazon.com - 36 млн книг). Для подобных задач характерен поиск по большому количеству критериев (около ста для одного предложения Яндекс.Маркета, двадцати - для amazon.com) и разнообразные принципы поиска по критериям (поиск по диапазону, по совпадению, полнотекстовый поиск). Задача осложняется неравномерным распределением критериев в данных и потенциально большими размером результирующих выборок. При этом, к базе данных предъявляются высокие требования к производительности (как к нагрузке, так и к скорости получения результата). Стандартные реляционные базы данных не удовлетворяют таким требованиям, современные NoSQL решения так же не реализуют необходимую функциональность.
    \section{Требования к разрабатываемому компоненту.}
        Функциональные требования:
        \begin{enumerate}\itemsep1pt \parskip0pt \parsep0pt
            \item Планирование порядка выполнения запроса
            \item Учет сложности запросов к сложным хранилищам (полнотекстовый поиск и т.п.)
            \item Возможность подстройки оптимизатора под конкретное железо
            \item Эффективная реализация выполнения запросов при массированных изменениях данных
            \item Реализация кэширования запросов к хранилищам с низкой производительностью
        \end{enumerate}

        Формальные показатели: %ничего не понятно (1)
        \begin{enumerate}\itemsep1pt \parskip0pt \parsep0pt
            \item Увеличение скорости выполнения отдельных запросов в два и более раз
        \end{enumerate}

        Нефункциональные требования: %ничего не понятно (2)
        \begin{enumerate}\itemsep1pt \parskip0pt \parsep0pt
            \item Низкие накладные расходы на обеспечение эффективности выполнения запросов
            \item Тестирование правильности функционирования выполнения запросов
        \end{enumerate}
    \section{Руководитель. В каком объеме предоставлена информация для выполнения ВКР.}
    	Научный руководитель и заказчик системы - Дельгядо Филипп Игоревич, куратор академических программ, JetBrains. Имеется возможность непосредственного взаимодействия, техническая документация Oracle Java и стандарты кодирования Java Coding Conventions
    \section{Описание технологий, которые будут использованы при разработке.}
        Язык реализации - Java. Некоторые фреймворки и библиотеки - Spring, JUnit, JProfiler, Apache Lucene, Java EWAH. Система контроля версий - Git. Репозиторий - github.com
        %в связи с модульностью виндура можно говорить о сравнении различных версий экзекьюторов на одних и тех же данных и разных версий движка, etc.
    \section{Описание методов тестированя и оценки качества разработанного компонента.}
        Тестирование - Performance, Unit testing. Так же, для оценки качества оптимизатора и базы в целом предполагается использование генератора тестовых данных и реальных данных, взятых из открытого источника.
    \section{Описание результатов работы.}
        \begin{enumerate}\itemsep1pt \parskip0pt \parsep0pt
        	\item Артефакты проектирования:
            	\begin{enumerate}\itemsep1pt \parskip0pt \parsep0pt
            	    \item Описание различных подходов к выполнению запросов
            	    \item Диаграммы UML программной архитектуры системы в целом
            	\end{enumerate}
        	\item Артефакты разработки:
            	\begin{enumerate}\itemsep1pt \parskip0pt \parsep0pt
            	    \item Исходный код в публичном репозитории
            	    \item Тестовые данные
            	\end{enumerate}
    	    \item Документация по системе
    	    \item Сравнение различных подходов к исполнению запросов на одних и тех же тестовых данных
        \end{enumerate}



        (см. статью ``C-store: a column-oriented DBMS.'' Mike Stonebraker, Daniel J. Abadi, Adam Batkin, Xuedong Chen, Mitch Cherniack, Miguel Ferreira, Edmond Lau, Amerson Lin, Sam Madden, Elizabeth O'Neil, Pat O'Neil, Alex Rasin, Nga Tran, Stan Zdonik. August 2005. VLDB '05: Proceedings of the 31st international conference on Very large data bases)

\end{document}
