% Тут используется класс, установленный на сервере Papeeria. На случай, если
% текст понадобится редактировать где-то в другом месте, рядом лежит файл matmex-diploma-custom.cls
% который в момент своего создания был идентичен классу, установленному на сервере.
% Для того, чтобы им воспользоваться, замените matmex-diploma на matmex-diploma-custom
% Если вы работаете исключительно в Papeeria то мы настоятельно рекомендуем пользоваться
% классом matmex-diploma, поскольку он будет автоматически обновляться по мере внесения корректив
%
\documentclass{matmex-diploma}
\usepackage{amssymb}
\usepackage{float}
\usepackage{algorithm}
\usepackage{algorithmic}

\floatname{algorithm}{Алгоритм}
\renewcommand{\algorithmicrequire}{\textbf{Входные параметры:}}
\renewcommand{\algorithmicensure}{\textbf{Результат:}}

\begin{document}
% Год, город, название университета и факультета предопределены,
% но можно и поменять.
% Если англоязычная титульная страница не нужна, то ее можно просто удалить.
\filltitle{ru}{
    chair              = {Кафедра Информационных Систем},
    title              = {Реализация эффективного выполнения поисковых запросов по множеству полей для колоночно-ориентированной базы данных, хранящейся в памяти.},
    % Здесь указывается тип работы. Возможные значения:
    %   coursework - Курсовая работа
    %   diploma - Диплом специалиста
    %   master - Диплом магистра
    %   bachelor - Диплом бакалавра
    type               = {bachelor},
    position           = {студента},
    group              = 4501,
    author             = {Жаворонков Эдгар Андреевич},
    %supervisorPosition = {д.\,ф.-м.\,н., профессор},
    supervisor         = {Дельгядо Ф.\,И.},
    reviewerPosition   = {к.\,т.\,н., доцент каф. КОТ},
    reviewer           = {Белозубов А.\,В.},
    chairHeadPosition  = {д.\,т.\,н., профессор},
    chairHead          = {Парфенов В.\,Г.},
    university         = {Санкт-Петербургский Национальный Исследовательский Университет Информационных Технологий, Механики и Оптики},
    faculty            = {Факультет Информационных Технологий и Программирования},
    city               = {Санкт-Петербург},
    year               = {2015}
}
\maketitle
\tableofcontents
% У введения нет номера главы
\section*{Введение}
    \subsection{Краткая история и архитектура колоночно-ориентированных баз данных}
        \paragraph{Историческая справка}
            Колоночно-ориентированные системы управления базами данных берут свое начало в 70-ых годах прошлого столетия. Так называемая TOD (Time Oriented Database) была предназначена для управления медицинскими записями. Одна из ранних систем, напоминающих современные колоночно-ориентированные СУБД была Cantor, которая, кроме всего прочего, включала в себя обильное использование различных техник сжатия данных (RLE - Run Length Encoding, сжатие числовых данных и т.д). Дальнейшее развитие привело к появлению на свет Sybase IQ, показавшей преимущества колоночно-ориентированного хранения данных для аналитических приложений. В середине 2000-ых годов, в связи с увеличением производительности аппаратного обеспечения, произошел резкий скачок в развитии колоночно-ориентированных хранилищ. На свет появились Vertica, Ingres, VectorWise, Paraccel и др. В дальнейшем, это привело к тому, что многие большие разработчики традиционных реляционных БД (такие как Oracle, Microsoft, IBM и др.) добавили поддержку хранения данных по колонкам - \cite{abadidesign}.
        \paragraph{Краткое описание архитектуры}
            Главное отличие колоночно-ориентированных баз данных, от традиционных - это физический способ хранения данных. Традиционные базы хранятся построчно, то есть все строки таблицы представлены в виде одной большой физической записи, в которой поля идут одно за другим.
        Плюсы такого подхода к хранению данных:
        \begin{enumerate}\itemsep1pt \parskip0pt \parsep0pt
            \item Быстрое добавление новых строк в таблицу
            \item Быстрая выборка строки по ключу
            \item Быстрая выборка всех строк
        \end{enumerate}
        Минусы:
        \begin{enumerate}\itemsep1pt \parskip0pt \parsep0pt
            \item Большое количество различных индексов, журналов - избыточность информации
            \item Медленная выборка нескольких колонок
            \item Медленная вставка новой колонки
            \item Избыточное хранение данных для полупустых колонок
        \end{enumerate}
        Рассмотрим подробнее второй недостаток. Представим себе базу данных, состоящую из одной таблицы на 50 колонок. Предположим, что нам нужно выбрать из нее лишь три. Но тогда, так как она хранится построчно, то значения всех полей будут считаны с диска, затем ненужные данные будут отброшены. Получение этой выборки произойдет за время равное $O(m * n)$, где $m$ - это количество строк, а $n$ - количество колонок. При большом количестве строк в такой базе, получение выборки по нескольким критериям - довольно дорогая операция, особенно с учетом ограничений дисковых интерфейсов ввода-вывода - \cite{habr:column_db}.
        
        \begin{figure}[h]
            \label{row_store}
            \centering
            \includegraphics[width=\textwidth]{../pics/row_store.png}
            \caption{Чтение всех столбцов из строчной БД}
        \end{figure}
        
        \begin{figure}[h]
            \label{column_store}
            \centering
            \includegraphics[width=\textwidth]{../pics/column_store.png}
            \caption{Чтение всех столбцов из колоночной БД}
        \end{figure}
        
        Колоночно-ориентированные базы данных хранят записи по колонкам, друг за другом.  Очевидны плюсы такого формата хранения данных:
        \begin{enumerate}\itemsep1pt \parskip0pt \parsep0pt
            \item Возможность сжатия данных
            \item Быстрая вставка новой колонки или изменение значений в существующей
            \item Быстрая выборка нескольких колонок
        \end{enumerate}
        Но есть и минус:
        \begin{enumerate}\itemsep1pt \parskip0pt \parsep0pt
            \item Изменение нескольких значений в строке происходит медленнее, чем у традиционных БД
        \end{enumerate}
    \paragraph{Решаемые задачи} 
        Из плюсов и минусов колоночно-ориентированных баз напрямую следуют задачи, которые они призваны решать. Как я уже упоминал в истории развития, такой подход к хранению данных хорош для аналитических приложений, в которых основная нагрузка создается выборкой больших объемов данных по нескольким критериям(как правило, 7 или 8). Либо же, задача многокритериального поиска, в которой, опять же, основная нагрузка - выборка большого объема данных по нескольким полям, но без какой-либо аналитики.
        \begin{figure}[h]
            \label{architecture_diff}
            \centering
            \includegraphics[width=\textwidth]{../pics/column-oriented-database1.jpg}
            \caption{Различия в способе хранения данных в традиционных и колоночно-ориентированных БД}
        \end{figure}
    \subsection{Постановка задачи}
        Целью данной работы является разработка компонента для базы данных, предназначенной для эффективной реализации поиска по большому количеству критериев. Подобная база данных может использоваться для большого количества разнообразных задач, в первую очередь для организации поиска по товарам в больших торговых площадках (wikimart.ru - 1.8 млн товаров, Яндекс.Маркет - 57 млн. предложений, amazon.com - 36 млн книг). Для подобных задач характерен поиск по большому количеству критериев (около ста для одного предложения Яндекс.Маркета, двадцати - для amazon.com) и разнообразные принципы поиска по критериям (поиск по диапазону, по совпадению, полнотекстовый поиск). Задача осложняется неравномерным распределением критериев в данных и потенциально большими размером результирующих выборок. При этом, к базе данных предъявляются высокие требования к производительности (как к нагрузке, так и к скорости получения результата) а так же - необходимость обеспечения целостности данных при массовых изменениях значения одного или нескольких критериев.
    \subsection{Требования к разрабатываемому решению}
        \subsubsection{Функциональные требования}
            \begin{enumerate}
                \item Планирование порядка выполнения запроса
                \item Учет сложности запросов к сложным хранилищам (полнотекстовый поиск и т.п.)
                \item Возможность автоконфигурирования исполнителя запросов
                \item Реализация корректного выполнения запросов на массированных изменениях данных
            \end{enumerate}
        \subsubsection{Формальные показатели}
            \begin{enumerate}\itemsep1pt \parskip0pt \parsep0pt 
                \item Уменьшение времени выполнения отдельных запросов в два и более раз
            \end{enumerate}
        \subsubsection{Нефункциональные требования}
            \begin{enumerate}\itemsep1pt \parskip0pt \parsep0pt 
                    \item Низкие накладные расходы на обеспечение эффективности выполнения запросов
                    \item Тестирование правильности функционирования выполнения запросов
            \end{enumerate}
    \subsection{Обзор аналогов}
        Ближайшие по архитектуре СУБД - это TimesTen by Oracle, VoltDB, SAP HANA. Все три используют хранение данных по колонкам, являются in-memory и предназначены для аналитических задач. Однако, их объединяет один общий недостаток - цена. Полные версии всех трех СУБД стоят больше десяти тысяч USD.
    
    Существует бесплатный вариант in-memory SQL базы данных - H2. Однако, он не колоночно-ориентирован и медленно работает на сложных операциях JOIN.
    
    Два решения, пригодных для решения задачи многокритериального поиска - это MySQL и Lucene. Вообще говоря, основное предназначение Apache Lucene - это полнотекстовый поиск, однако им можно пользоваться и как in-memory БД для поиска по нескольким критериям. Недостаток Lucene - очень медленная операция commit. Недостатки MySQL - проблемы совместимости различных движков (InnoDB и MyISAM) и большое время поиска по нескольким индексам с последующим пересечением результатов. 
    
    \begin{table}[h]
        \centering
        \begin{tabular}{| l | c | c | c | c |}
            \hline
            Product & in-memory & NoSQL & OLAP & Price \\
            \hline
            Oracle TimesTen & \color{green}{\checkmark} & \color{green}{\checkmark} & \color{green}{\checkmark} & \color{red}{€19,969.00} \\
            \hline 
            VoltDB          & \color{green}{\checkmark} & \color{green}{\checkmark} & \color{green}{\checkmark} & \color{red}{\$3500/month} \\
            \hline
            SAP HANA        & \color{green}{\checkmark} & \color{green}{\checkmark} & \color{green}{\checkmark} & \color{red}{\$3595/month}\\
            \hline
        \end{tabular}
        \caption{Сравнительная характеристика TimesTen, VoltDB и SAP HANA}
    \end{table}
    
\section{Описание проектных решений}
    %или программной?
    \subsection{Описание системной архитектуры модифицируемого решения}
    Система управления базами данных Vindur представляет собой легко расширяемую встраиваемую NoSQL базу данных, имеющую колоночно-ориентированную архитектуру, хранящую данные в оперативной памяти. Задача, решаемая с помощью Vindur - это быстрый поиск по многим критериям. 
    Основные сущности и компоненты Vindur:
        \begin{enumerate}
            \item \textbf{Документ}(Document) - основная сущность, хранящаяся в базе данных. Представляет собой набор атрибутов и соответствующих значений. Каждый документ обладает уникальным идентификационным номером.                 
            \item \textbf{Хранилище}(Storage) - структура данных, хранящая в том или ином виде значения отдельно взятого атрибута для всех документов. Поддерживает операции поиска и изменения значения в отдельно взятом документе. Также поддерживает операцию поиска всех документов с заданным значением атрибута.
            \item \textbf{Запрос}(Query) - набор критериев к документу, которым он должен соответствовать для включения в результирующую выборку. Формально - множество пар $(a, v)$, где $a$ - атрибут, а $v$ - его значение. 
            \item \textbf{Результат запроса}(Query result) - множество документов, удовлетворяющих данному запросу-выборке.
            \item \textbf{Вес хранилища}(Complexity) - эмпирическая оценка относительной трудоемкости запроса к данному хранилищу.
            \item \textbf{Движок базы данных}(Engine) - главный компонент всей системы, хранящий в себе документы, хранилища и содержащий методы для добавления нового документа, изменения значения выбранного атрибута в отдельно взятом документе, а также - выполнения запросов-выборок.
            \item \textbf{Исполнитель}(Executor) - компонент базы данных, осуществляющий исполнение поисковых запросов.
        \end{enumerate}
    
    Основной единицей хранения в Vindur является документ - набор полей(атрибутов) и соответствующих значений атрибутов. Каждый атрибут может иметь одно или несколько значений. Кроме того, документ обладает уникальным идентификационным номером, по которому движок базы данных обращается к нему. 
    
    Значения каждого атрибута хранятся независимо в специально предназначенных хранилищах. Каждое хранилище ориентировано для выполнения определенного вида запросов(поиск по точному совпадению, поиск на попадание в диапазон, полнотекстовый поиск) к нему. Иначе говоря, для каждого атрибута можно указать только один тип критерия, на который он будет проверяться. На данный момент реализованы критерии "точное соответствие", "включен в диапазон", "соответствует маске", "входит в иерархию" 
    
    По своей структуре, хранилище представляет собой набор битовых масок. Результат запроса к хранилищу - битовая маска, единичный бит в $i$-ой позиции означает, что в документе с номером $i$ значение данного атрибута соответствует запросу. Битовые маски используются по ряду причин, основная из которых - возможность сжатия. Кроме того, операции пересечения множеств документов, удовлетворяющих различным критериям соответствует операция логического "И", выполненная для соответствующих битовых масок. 
    
    Рассмотрим процесс выполнения поискового запроса к базе данных Vindur. Он состоит из двух этапов - проверки на корректность(для всех атрибутов проверяется, можно ли к данному хранилищу выполнить запрос такого типа) и непосредственно выполнения. Выполнение заключается в последовательном получении битовых масок из хранилищ с последующим пересечением. На данный момент реализованы три стратегии выполнения поисковых запросов:
    \begin{enumerate}
        \item Тривиальная стратегия (так называемый алгоритм DumbExecutor)
        \item Стратегия выполнения запроса с весами хранилищ (алгоритм SmartExecutor)
        \item Стратегия выполнения запросов на основе статистики по атрибутам (алгоритм TunableExecutor)
    \end{enumerate}
    
    \begin{figure}[H]
        \label{query}
        \centering
        \includegraphics[width=0.5\textwidth]{../pics/query.png}
        \caption{Общая схема выполнения поискового запроса в СУБД Vindur}
    \end{figure}
    %а нужен ли?????
    %\pagebreak
    \subsection{Описание тривиального исполнителя запросов}
        Алгоритм DumbExecutor принимает на вход поисковый запрос $Q$, разбивает его на части $\left\{ q_1, q_2, q_3, ... , q_n \right\} $ и для каждой части выполняет операцию обращения к хранилищу, содержащему данные по данному атрибуту. Выход алгоритма - список номеров документов, удовлетворяющих все критериям в запросе. Псевдокод алгоритма выглядит следующим образом.
        \begin{algorithm}                   
        \caption{DumbExecutor}              
        \label{dumb}                        
            \begin{algorithmic}        
                \REQUIRE $Q$ - запрос
                \ENSURE $resultSet$ - множество документов, удовлетворяющих запросу $Q$
                \STATE $\left\{ q_1, q_2, q_3, ... , q_n \right\} \leftarrow SplitIntoParts(Q);$
                \STATE $resultSet \leftarrow NULL;$
                \FOR{$ i \in \{1..n\}$}
                    \STATE $a \leftarrow Attribute(q_i);$
                    \STATE $v \leftarrow Value(q_i);$
                    \STATE $stepResult \leftarrow FindSet(a, v);$
                    \IF{$resultSet == NULL$}
                        \STATE $resultSet \leftarrow stepResult;$
                    \ELSE
                        \STATE $resultSet \leftarrow resultSet \land stepResult;$
                    \ENDIF
                \ENDFOR
                \RETURN $ConvertToIntList(resultSet);$
            \end{algorithmic}
        \end{algorithm}
        
        Функция $SplitIntoParts$ принимает на вход запрос $Q$ и возвращает список всех частей запроса. Функции $Attribute$ и $Value$ принимают на вход часть запроса $q$ и возвращают соответственно атрибут и значение, хранящиеся в данной части запроса. Функция $findSet$ принимает на вход атрибут и значение и возвращает битовую маску - результат обращения к соответствующему хранилищу. В данном случае $resultSet $ и $stepResult$ - это битовые маски. Функция $СonvertToIntList$ принимает на вход битовую маску $mask$ и возвращает список позиций, таких что $mask_i = 1$ 
        
    \subsection{Описание исполнителя запросов с весами}
        Алгоритм SmartExecutor принимает на вход поисковый запрос $Q$, разбивает его на части $\left\{ q_1, q_2, q_3, ... , q_n \right\} $ и ставит в соответствие каждому атрибуту из частей запроса вес хранилища. После этого, он обращается к хранилищам в порядке увеличения весов хранилищ. Кроме запроса, алгоритм принимает на вход размер результирующей выборки, начиная с которого проверка документов на соответствие критериям запроса происходит в обход хранилищ - так называемый порог срабатывания. Выход алгоритма - так же список номеров документов, удовлетворяющих все критериям в запросе. В данном случае веса хранилищ устанавливаются эмпирически. Псевдокод выглядит следующим образом:
        
        \begin{algorithm}[H]                   
        \caption{SmartExecutor}              
        \label{smart}                        
            \begin{algorithmic}        
                \REQUIRE $Q$ - запрос, $threshold$ - порог срабатывания
                \ENSURE $resultSet$ - множество документов, удовлетворяющих запросу $Q$
                \STATE $\left\{ q_1, q_2, q_3, ... , q_n \right\} \leftarrow SplitIntoParts(Q);$
                \STATE $Sort(\left\{ q_1, q_2, q_3, ... , q_n \right\}, Compare);$
                \STATE $resultSet \leftarrow NULL;$
                \FOR{$ i \in \{1..n\}$}
                    \STATE $a \leftarrow Attribute(q_i);$
                    \STATE $v \leftarrow Value(q_i);$
                    \STATE $stepResult \leftarrow FindSet(a, v);$
                    
                    \IF{$resultSet == NULL$}
                        \STATE $resultSet \leftarrow stepResult;$
                    \ELSE
                        \STATE $resultSet \leftarrow resultSet \land stepResult;$
                    \ENDIF
                    
                    \IF{$Cardinality(resultSet) == 0$}
                        \RETURN $EMPTY\_LIST;$
                    \ENDIF
                    
                    \IF{$Cardinality(resultSet) \leq threshold$}
                        \STATE $tail \leftarrow \left\{ q_1, q_2, q_3, ... , q_n \right\} \setminus \left\{ q_1, q_2, q_3, ... , q_i \right\};$
                        \RETURN $CheckManually(tail, resultSet);$
                    \ENDIF
                \ENDFOR
                \RETURN $ConvertToIntList(resultSet);$
            \end{algorithmic}
        \end{algorithm}
        
        Функция $Sort$ принимает на вход части $\left\{ q_1, q_2, q_3, ... , q_n \right\}$ запроса $Q$ и сортирует их по убыванию весов соответствующих хранилищ. Для этого, вторым параметром она принимает компаратор - функцию сравнения двух частей запроса, которая возвращает $-1$, если хранилище для атрибута $a_1$ имеет меньший вес, чем хранилище для атрибута $a_2$, $0$ - если веса хранилищ равны, $1$ - в противном случае. Функция $Complexity$ принимает на вход атрибут и возвращает вес хранилища, ассоциированного с данным атрибутом. Функция $Cardinality$ принимает на вход битовую маску и возвращает количество единичных бит в ней.
        
        \begin{algorithm}[H]                   
        \caption{Compare}              
        \label{cmp}                        
            \begin{algorithmic}        
                \REQUIRE $q_1$, $q_2$ - части запроса
                \ENSURE Результат сравнения по весам хранилищ двух частей запроса
                \STATE $a_1 \leftarrow Attribute(q_1);$
                \STATE $a_2 \leftarrow Attribute(q_2);$
                \IF {$Complexity(a_1) < Compexity(a_2)$}
                    \RETURN $-1;$
                \ELSE 
                    \IF {$Complexity(a_1) = Compexity(a_2)$}
                        \RETURN $0;$
                    \ELSE
                        \RETURN $1;$
                    \ENDIF
                \ENDIF
            \end{algorithmic}
        \end{algorithm}
        
        Функция $CheckManually$ осуществляет проверку документов на соответствие оставшимся критериям в запросе $Q$ в обход хранилищ. На вход подается текущая результирующая выборка и оставшиеся части $\left\{ q_1, q_2, q_3, ... , q_n \right\} \setminus \left\{ q_1, q_2, q_3, ... , q_i \right\}$ запроса $Q$. На выходе - cписок номеров документов, удовлетворяющим всем критериям из запроса $Q$.   
        
        \begin{algorithm}[H]                   
        \caption{CheckManually}              
        \label{check}                        
            \begin{algorithmic}        
                \REQUIRE $tail$ - оставшиеся части запроса, $resultSet$ - результирующая выборка
                \ENSURE $resultSet'$ - множество документов, удовлетворяющих оставшимся частям запроса $tail$
                \STATE $resultSet' \leftarrow EMPTY\_SET;$
                \FOR {$q \in tail$}
                    \FOR {$docID \in ConvertToIntList(resultSet)$}
                        \STATE $a \leftarrow Attribute(q)$, $v \leftarrow Value(q);$
                        \STATE $values \leftarrow getValues(getDocument(docID), a);$
                        \FOR {$value \in values$}
                            \IF {$v \in values$}
                                \STATE $resultSet'_{docID} \leftarrow 1;$
                            \ENDIF
                        \ENDFOR
                    \ENDFOR
                \ENDFOR
                \STATE $resultSet' \leftarrow resultSet \land resultSet';$
                \RETURN $ConverToIntList(resultSet');$
            \end{algorithmic}
        \end{algorithm}
        
    \subsection{Описание автоматически настраивающегося исполнителя}
        %рассказать про дополнительный поток в движке, про тюнер, про очередь запросов, затем рассказать про журнал статистики и потом только про исполнение 
        Алгоритм TunableExecutor использует параллельный поток исполнения, в котором считается среднее время выполнения запроса к хранилищу. Рассмотрим подробнее его работу.
        
        Еще одна сущность Vindur - это так называемый \textbf{настройщик}(Tuner). Это компонент, собирающий статистические данные для всех атрибутов. Статистические данные включают в себя среднее время выполнения операции $findSet$  к хранилищу и среднее время проверки значения атрибута в документе напрямую.
        
        При инциализации движка базы данных происходит создание и запуск параллельного потока, который в бесконечном цикле вызывает операцию $callTuner$ один раз в секунду.
        
        При исполнении поискового запроса перед проверкой на корректность происходит добавление запроса в очередь. Настройщик извлекает последний запрос, попавший в очередь и для кадого атрибута в запросе измеряет среднее время выполнения операции $findSet$  к хранилищу и среднее время проверки значения атрибута в документе напрямую. Результаты сохраняются в специальный журнал, доступный как и потоку, в котором работает настройщик, так и потоку, в котором работает исполнитель запросов.
        
        Соответственно, алгоритм TunableExecutor принимает на вход поисковый запрос $Q$, разбивает его на части $\left\{ q_1, q_2, q_3, ... , q_n \right\} $ и ставит в соответствие каждому атрибуту из частей запроса среднее время выполнения операции $findSet$, если же по каким-либо причинам статистических данных нет, то в соответствие атрибуту ставится вес хранилища. После этого, он обращается к хранилищам в порядке увеличения среднего времени выполнения. 
        
        Кроме того, алгоритм на каждом шаге считает, какое время займет ручная проверка оставшихся критериев в запросе. Если получится так, что проверка в обход хранилищ окажется быстрее, то алгоритм проверит текущую результирующую выборку на соответствие остальным критериям в запросе напрямую. Выход алгоритма - так же список номеров документов, удовлетворяющих все критериям в запросе. Псевдокод выглядит следующим образом:
        
        \begin{algorithm}[H]
        \caption{TunerThread}
        \label{thread}
            \begin{algorithmic}
                \WHILE {TRUE}
                    \STATE $callTuner();$
                    \STATE $Sleep(1000);$
                \ENDWHILE
            \end{algorithmic}
        \end{algorithm}
        
        Функция $CallTuner$ проверяет, не пуста ли очередь запросов $queries$, если она не пуста - она извлекает из очереди запрос и для каждого атрибута в запросе считает среднее время выполнения операции $findSet$ и прямой проверки, записывая данные в специальный журнал $journal$   
        
        \begin{algorithm}[H]
        \caption{InitEngine}
        \label{init}
            \begin{algorithmic}
                \STATE $StartThread(TunerThread);$
            \end{algorithmic}
        \end{algorithm}
        
        Инициализация движка Vindur внутри которой, происходит запуск потока $TunerThread$.
        
        \begin{algorithm}[H]
        \caption{ExecuteQuery}
        \label{exec}
            \begin{algorithmic}
                \REQUIRE $Q$ - поисковый запрос
                \ENSURE $resultSet$ - множество документов, удовлетворяющих запросу $Q$
                \STATE $push(Q, queries);$
                \STATE $CheckQuery(Q);$  
                \RETURN $TunableExecutor(Q);$
            \end{algorithmic}
        \end{algorithm}
        
        Полный алгоритм выполнения поискового запроса с проверкой на корректность и добавлением запроса $Q$ в очередь $queries$ 
        
        \begin{algorithm}[H]                   
        \caption{TunableExecutor}              
        \label{tunable}                        
            \begin{algorithmic}        
                \REQUIRE $Q$ - запрос
                \ENSURE $resultSet$ - множество документов, удовлетворяющих запросу $Q$
                \STATE $\left\{ q_1, q_2, q_3, ... , q_n \right\} \leftarrow SplitIntoParts(Q);$
                \STATE $journal \leftarrow GetJournal();$
                \STATE $Sort(\left\{ q_1, q_2, q_3, ... , q_n \right\}, Compare);$
                \STATE $resultSet \leftarrow NULL;$
                
                \STATE $estimatedExecutionTime \leftarrow 0;$
                \FOR {$ i \in \{1..n\} $}
                    \STATE $a \leftarrow Attribute(q_i);$
                    \STATE $estimatedExecutionTime = estimatedExecutionTime + ExecutionTime(journal, a);$
                \ENDFOR
                
                \FOR {$ i \in \{1..n\} $}
                    \STATE $a \leftarrow Attribute(q_i);$
                    \STATE $v \leftarrow Value(q_i);$
                    \STATE $stepResult \leftarrow FindSet(a, v);$
                    \STATE $estimatedExecutionTime \leftarrow estimatedExecutionTime - ExecutionTime(journal, a);$
                    \IF{$resultSet == NULL$}
                        \STATE $resultSet \leftarrow stepResult;$
                    \ELSE
                        \STATE $resultSet \leftarrow resultSet \land stepResult;$
                    \ENDIF
                    \STATE $partsLeft \leftarrow n - i;$
                    \IF{$Cardinality(resultSet) == 0$}
                        \RETURN $EMPTY\_LIST;$
                    \ENDIF
                    \STATE $estimatedCheckTime \leftarrow partsLeft * CheckTime(journal, a);$
                    \IF{$estimatedCheckTime \leq estimatedExecutionTime$}
                        \STATE $tail \leftarrow \left\{ q_1, q_2, q_3, ... , q_n \right\} \setminus \left\{ q_1, q_2, q_3, ... , q_i \right\}$
                        \RETURN $CheckManually(tail, resultSet);$
                    \ENDIF
                \ENDFOR
                \RETURN $ConvertToIntList(resultSet;)$
            \end{algorithmic}
        \end{algorithm}
        
        Функция $GetJournal$ возвращает журнал, содержащий статистические данные для атрибутов, по которым настройщик измерял среднее время выполнения. Функции $ExecutionTime$ и $CheckTime$ принимают на вход журнал статистики и атрибут и возвращают среднее время обращения к хранилищу и среднее время проверки значения в документе соответственно. Функция $Compare$ точно так же осуществляет сортировку множества частей запроса $Q$ по возрастанию среднего времени обращения к соответствующим хранилищам.   
        
        \begin{algorithm}[H]                   
        \caption{Compare}              
        \label{cmp1}                        
            \begin{algorithmic}        
                \REQUIRE $q_1$, $q_2$ - части запроса $journal$ - журнал, хранящий статистические данные по атрибутам
                \ENSURE Результат сравнения двух частей запроса
                \STATE $a_1 \leftarrow Attribute(q_1)$
                \STATE $a_2 \leftarrow Attribute(q_2)$
                \IF {$ExecutionTime(journal, a_1) < ExecutionTime(journal, a_2)$}
                    \RETURN $-1$
                \ELSE 
                    \IF {$ExecutionTime(journal, a_1) = ExecutionTime(journal, a_2)$}
                        \RETURN $0;$
                    \ELSE
                        \RETURN $1;$
                    \ENDIF
                \ENDIF
            \end{algorithmic}
        \end{algorithm}
        
    \subsection{Описание первого варианта реализации атомарного добавления документа}
    \subsection{Описание второго варианта реализации атомарного добавления документа}
    \subsection{Описание тестовой инфраструктуры}
    
\section{Особенности реализации}
    \subsection{Описание технологической платформы}
    \subsection{Пример реализации SmartExecutor}
    \subsection{Реализация тестовой инфраструктуры}
    \subsection{Результаты тестирования}
    
\section*{Заключение}

\bibliographystyle{gost780s}
\bibliography{diploma.bib}
\end{document}
