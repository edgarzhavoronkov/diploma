%% Простая презентация с примером включения программного кода и
%% пошаговых спецэффектов
\documentclass{beamer}
\usepackage{fontspec}
\usepackage{xunicode}
\usepackage{xltxtra}
\usepackage{xecyr}
\usepackage{hyperref}
\setmainfont[Mapping=tex-text]{DejaVu Serif}
\setsansfont[Mapping=tex-text]{DejaVu Sans}
\setmonofont[Mapping=tex-text]{DejaVu Sans Mono}
\usepackage{polyglossia}
\setdefaultlanguage{russian}
\usepackage{graphicx}
\usepackage{listings}
\usepackage{amssymb}% http://ctan.org/pkg/amssymb
\usepackage{pifont}
\usetheme{CambridgeUS}

\lstdefinestyle{mycode}{
  belowcaptionskip=1\baselineskip,
  breaklines=true,
  xleftmargin=\parindent,
  showstringspaces=false,
  basicstyle=\footnotesize\ttfamily,
  keywordstyle=\bfseries,
  commentstyle=\itshape\color{gray!40!black},
  stringstyle=\color{red},
  numbers=left,
  numbersep=5pt,
  numberstyle=\tiny\color{gray},
}
\lstset{escapechar=@,style=mycode}

\begin{document}
\title{Реализация эффективного выполнения поисковых запросов по множеству полей для колоночно-ориентированной базы данных, хранящейся в памяти.}
\subtitle{предварительные результаты}
\author{Жаворонков Эдгар Андреевич \\ группа 4501 \\ руководитель Дельгядо Ф.И}
\institute{Университет ИТМО}

\frame{\titlepage}

\begin{frame}\frametitle{Актуальность разработки}
    \begin{enumerate}
        \item Создание Open Source in-memory NoSQL базы данных, позволяющей решать задачу быстрого многокритериального поиска
        \item Существующие решения либо очень дороги, либо ориентированы больше в сторону аналитических приложений
        \item <2-> Либо медленные, так как работают с диском
    \end{enumerate}
\end{frame}

\begin{frame}\frametitle{Обзор предметной области}
    \begin{enumerate}
        \item Текущее состояние дел таково, что в большинстве приложений требующих многокритериальный поиск(интернет-магазины, библиотеки) используются традиционные SQL решения (MySQL, PostgreSQL etc.) 
        \item Гиганты индустрии(Amazon.com, ebay.com) используют NoSQL-решения, но они, как правило, дороги.
    \end{enumerate}
\end{frame}

\begin{frame}\frametitle{Аналоги}
    \begin{tabular}{| l | c | c | c | c |}
        \hline
        Product & in-memory & NoSQL & OLAP & Price \\
        \hline
        Oracle TimesTen & \color{green}{\checkmark} & \color{green}{\checkmark} & \color{green}{\checkmark} & \color{red}{€19,969.00} \\
        \hline 
        VoltDB          & \color{green}{\checkmark} & \color{green}{\checkmark} & \color{green}{\checkmark} & \color{red}{\$3500/month} \\
        \hline
        SAP HANA        & \color{green}{\checkmark} & \color{green}{\checkmark} & \color{green}{\checkmark} & \color{red}{\$3595/month}\\
        \hline
    \end{tabular}
    \begin{enumerate}
        \item Все три решения - очень мощные аналитические инструменты, использующие поколоночное хранение данных в оперативной памяти. 
        \item Как видно, их основной недостаток - цена. Вендоры, кроме всего прочего, предоставляют сервера, на которых разворачиваются эти решения. 
    \end{enumerate}
\end{frame}

\begin{frame}\frametitle{Цель и задачи ВКР}
    Цель работы - разработать компонент, обеспечивающий быстрое выполнение поисковых запросов по нескольким полям для in-memory базы данных.
    
    Задачи:
    \begin{enumerate}
        \item Разобраться в существующей архитектуре
        \item Попробовать различные подходы к выполнению поисковых запросов
        \item Реализовать возможность обеспечения целостности данных на массовых загрузках документов
    \end{enumerate}
\end{frame}

\begin{frame}\frametitle{Требования}
    Функциональные требования:
        \begin{enumerate}\itemsep0pt \parskip0pt \parsep0pt
            \item Планирование порядка выполнения запроса
            \item Учет сложности запросов к сложным хранилищам (полнотекстовый поиск и т.п.)
            \item Возможность подстройки оптимизатора под конкретное железо
            \item Эффективная реализация выполнения запросов на массированных изменениях данных
        \end{enumerate}
    Формальные показатели:
        \begin{enumerate}\itemsep0pt \parskip0pt \parsep0pt 
            \item Увеличение времени выполнения отдельных запросов в два и более раз
        \end{enumerate}
    Нефункциональные требования:
        \begin{enumerate}\itemsep0pt \parskip0pt \parsep0pt 
                \item Низкие накладные расходы на обеспечение эффективности выполнения запросов
                \item Тестирование корректности выполнения запросов
        \end{enumerate}
\end{frame}

%сюда бы слайд с диаграммой классов....

\begin{frame}\frametitle{Архитектура системы в целом}
    \begin{enumerate}
        \item Основная сущность - документ с уникальным номером
        \item Значения атрибутов лежат в индексах - хранилищах. Хранилище - набор сжатых битовых множеств
        \item Каждое хранилище предназначено для выполнения запросов определенного типа
        \item Отдельный модуль, исполняющий запросы
    \end{enumerate}
\end{frame}

\begin{frame}\frametitle{Текущие результаты}
    \begin{enumerate}
        \item Различные способы исполнения запроса:
            \begin{enumerate}
                \item От начала и до конца, не трогая ничего
                \item Проверяя результат на непустоту 
                \item В порядке возрастания "сложности" получения данных из индекса
                \item Собирая статистику по каждому атрибуту
            \end{enumerate}
        \item Примитивный интерфейс транзакций
        \item Тесты производительности
    \end{enumerate}
\end{frame}

\begin{frame}\frametitle{Использованные инструменты}
    \begin{enumerate}
        \item Java 8
        \item Java EWAH
        \item Apache Lucene
        \item Git 
        \item Travis CI
    \end{enumerate}
\end{frame}

\begin{frame}\frametitle{Планируемые доработки}
    \begin{enumerate}
        \item Довести до конца тестирование компонента 
        \item Реализовать более производительный вариант транзакций
    \end{enumerate}
\end{frame}

\begin{frame} 
    \begin{center} 
        \Huge Спасибо за внимание! \\ Вопросы? 
    \end{center} 
\end{frame} 

\end{document}