%% Простая презентация с примером включения программного кода и
%% пошаговых спецэффектов
\documentclass{beamer}
\usepackage{fontspec}
\usepackage{xunicode}
\usepackage{xltxtra}
\usepackage{xecyr}
\usepackage{hyperref}
\setmainfont[Mapping=tex-text]{DejaVu Serif}
\setsansfont[Mapping=tex-text]{DejaVu Sans}
\setmonofont[Mapping=tex-text]{DejaVu Sans Mono}
\usepackage{polyglossia}
\setdefaultlanguage{russian}
\usepackage{graphicx}
\usepackage{listings}
\usepackage{amssymb}% http://ctan.org/pkg/amssymb
\usepackage{pifont}
\usepackage{tabulary}
\usepackage{tabularx}
\usetheme{CambridgeUS}

\lstdefinestyle{mycode}{
  belowcaptionskip=1\baselineskip,
  breaklines=true,
  xleftmargin=\parindent,
  showstringspaces=false,
  basicstyle=\footnotesize\ttfamily,
  keywordstyle=\bfseries,
  commentstyle=\itshape\color{gray!40!black},
  stringstyle=\color{red},
  numbers=left,
  numbersep=5pt,
  numberstyle=\tiny\color{gray},
}
\lstset{escapechar=@,style=mycode}

\begin{document}
\title{Реализация эффективного выполнения поисковых запросов по множеству полей для колоночно-ориентированной базы данных, хранящейся в памяти.}
\author{Жаворонков Эдгар Андреевич \\ группа 4501 \\ руководитель Дельгядо Ф.И}
\date{4 июня 2015}
\institute{Университет ИТМО}

\frame{\titlepage}

\begin{frame}\frametitle{Актуальность разработки}
    \begin{enumerate}
        \item Создание Open Source in-memory NoSQL базы данных, позволяющей решать задачу быстрого многокритериального поиска
        \item Существующие решения либо очень дороги, либо ориентированы больше в сторону аналитических приложений
        \item <2-> Либо медленные, так как работают с диском
    \end{enumerate}
\end{frame}

\begin{frame}\frametitle{Анализ предметной области}
    \begin{enumerate}
        \item Текущее состояние дел таково, что в большинстве приложений требующих многокритериальный поиск(интернет-магазины, библиотеки) используются традиционные SQL решения (MySQL, PostgreSQL etc.) 
        \item Гиганты индустрии(Amazon.com, ebay.com) используют собственные решения.
    \end{enumerate}
\end{frame}

\begin{frame}\frametitle{Обзор аналогов}
    \begin{tabular}{| l | c | c | c | c |}
        \hline
        Product & in-memory & NoSQL & OLAP & Price \\
        \hline
        Oracle TimesTen & \color{green}{\checkmark} & \color{green}{\checkmark} & \color{green}{\checkmark} & \color{red}{€19,969.00} \\
        \hline 
        VoltDB          & \color{green}{\checkmark} & \color{green}{\checkmark} & \color{green}{\checkmark} & \color{red}{\$3500/month} \\
        \hline
        SAP HANA        & \color{green}{\checkmark} & \color{green}{\checkmark} & \color{green}{\checkmark} & \color{red}{\$3595/month}\\
        \hline
    \end{tabular}
    \begin{enumerate}
        \item Все три решения - очень мощные аналитические инструменты, использующие поколоночное хранение данных в оперативной памяти. 
        \item Как видно, их основной недостаток - цена. Вендоры, кроме всего прочего, предоставляют сервера, на которых разворачиваются эти решения. 
    \end{enumerate}
\end{frame}

\begin{frame}\frametitle{Цель и задачи ВКР}
    Цель работы - разработать компонент, обеспечивающий быстрое выполнение поисковых запросов по нескольким полям для in-memory базы данных. Кроме того, требуется обеспечить возможность атомарного добавления записей в такую базу.
    
    Задачи:
    \begin{enumerate}
        \item Разработать и описать алгоритмы выполнения поисковых запросов;
        \item Реализовать алгоритмы;
        \item Ввести формальные показатели для сравнения и сравнить алгоритмы между собой; 
        \item Разработать, описать, реализовать между собой алгоритмы атомарного добавления записей;
    \end{enumerate}
\end{frame}

\begin{frame}\frametitle{Требования к решению}
    Функциональные требования:
        \begin{enumerate}\itemsep0pt \parskip0pt \parsep0pt
            \item Планирование порядка выполнения запроса;
            \item Учет сложности запросов к сложным хранилищам (полнотекстовый поиск и т.п.);
            \item Возможность автоматической подстройки модуля;
            \item Реализация корректного выполнения запросов на массированных изменениях данных;
        \end{enumerate}
    Формальные показатели:
        \begin{enumerate}\itemsep0pt \parskip0pt \parsep0pt 
            \item Уменьшение времени выполнения отдельных запросов в два и более раз
        \end{enumerate}
    Нефункциональные требования:
        \begin{enumerate}\itemsep0pt \parskip0pt \parsep0pt 
                \item Низкие накладные расходы на обеспечение эффективности выполнения запросов;
                \item Тестирование правильности выполнения запросов
        \end{enumerate}
\end{frame}

%сюда бы слайд с диаграммой классов....

\begin{frame}\frametitle{Архитектура системы. Общий обзор}
    \begin{enumerate}
        \item Основная сущность - документ с уникальным номером. Документ - набор полей(атрибутов) и их значений;
        \item Значения атрибутов лежат в индексах - хранилищах. Хранилище - набор сжатых битовых массивов;
        \item Каждое хранилище предназначено для выполнения запросов определенного типа;
        \item Отдельный модуль, исполняющий запросы;
    \end{enumerate}
\end{frame}

\begin{frame}\frametitle{Архитектура системы. Классы и сущности}
    \begin{figure}[H]
        \centering
        \includegraphics[height=0.7\textheight,keepaspectratio]{../pics/diagram1.png}
        \caption{Диаграмма классов системы Vindur.}
        \label{classes}
        \end{figure}
\end{frame}

\begin{frame}\frametitle{Специфика решения. Процесс выполнения запроса}
    Формально, запрос - это множество пар $(a, v)$, где $a$ - атрибут по которому осуществляется поиск, а $v$ - искомое значение.
    Нестрого говоря - несколько этапов:
    \begin{enumerate}
        \item Проверка на корректность;
        \item Получение битовых массивов из хранилищ;
        \item Пересечение между собой;
        \item Преобразование в список с номерами документов;
    \end{enumerate}
\end{frame}

\begin{frame}\frametitle{Специфика решения. Стратегии выполнения запросов}
    На текущий момент реализованы три стратегии:
    \begin{enumerate}
        \item Тривиальная (DumbExecutor);
        \item На основе весов хранилищ (SmartExecutor);
        \item На основе статистики по атрибутам (TunableExecutor);
    \end{enumerate}
\end{frame}

\begin{frame}\frametitle{Специфика решения. Тривиальная стратегия}
    \begin{enumerate}
        \item Для каждого атрибута, получаем ассоциированное с ним хранилище;
        \item Извлекаем оттуда битовый массив документов, имеющих соответствующее значение;
        \item Все битовые массивы "сворачиваются" операцией логического умножения(AND);
        \item Это соответствует пересечению множеств между собой; 
    \end{enumerate}
\end{frame}

\begin{frame}\frametitle{Специфика решения. Стратегия с весами}
    \textbf{Вес хранилища} - эмпирическая мера относительной трудоёмкости извлечения информации из хранилища.
    \begin{enumerate}
        \item Множество пар $(a, v)$ сортируется по возрастанию веса хранилищ, ассоциированных с атрибутами;
        \item Снова получаем хранилища извлекаем битовые массивы с документами и пересекаем их;
        \item Если на очередном шаге получили пустой битовый массив - прекратили выполнение;
        \item Если в процессе выполнения результирующая выборка стала содержать мало документов, то оставшиеся части запроса проверяются напрямую, минуя хранилища;
    \end{enumerate}
\end{frame}

\begin{frame}\frametitle{Специфика решения. Стратегия со статистикой}
    Вместо весов теперь - среднее время обращения к хранилищу и среднее время проверки в обход него.
    \begin{enumerate}
        \item Преподсчитывается \textit{примерное} время выполнения запроса и время проверки одного документа;
        \item Множество пар $(a, v)$ сортируется по возрастанию среднего времени обращения к хранилищам, ассоциированных с атрибутами;
        \item Если в процессе выполнения время проверки оставшихся частей в запросе для результирующей выборки стало меньше, чем время обращения к оставшимся хранилищам, то оставшиеся части запроса проверяются напрямую;
    \end{enumerate}
\end{frame}

\begin{frame}\frametitle{Специфика решения. Атомарное добавление 1}
    \begin{enumerate}
        \item "Добавляем - не ищем. Ищем - не добавляем";
        \item В ходе сеанса все изменения записываются в журнал;
        \item По подтверждению - БД блокируется и все изменения фиксируются;
        \item Блокировка означает, что ни один параллельный поток не имеет в это время доступа к базе;
    \end{enumerate}
\end{frame}

\begin{frame}\frametitle{Специфика решения. Атомарное добавление 2}
    \begin{enumerate}
        \item Изменения снова записываются в журнал;
        \item Однако по подтверждению - не применяются, а начинают учитываться при поиске документов(записываются в ещё один журнал); 
        \item Применение происходит небольшими порциями, параллельно выполнению запроса;
    \end{enumerate}
\end{frame}

\begin{frame}\frametitle{Тестирование}
    Задачи тестирования:
    \begin{enumerate}
        \item Определить среднее время выполнения запроса для разных алгоритмов; 
        \item Выявить достоинства и недостатки алгоритмов атомарного добавления записей;
    \end{enumerate}
\end{frame}

\begin{frame}\frametitle{Результаты тестирования алгоритмов выполнения запросов}
    Один миллион документов, три атрибута, пять тысяч запросов
    \begin{table}[H]
            \caption{Результаты тестирования алгоритмов выполнения поисковых запросов}
            \centering
                \begin{tabularx}{\textwidth}{| l | X | c | X |}
                    \hline
                    Алгоритм  & Среднее время выполнения запроса  & Дисперсия & Стандартное отклонение  \\
                    \hline
                    DumbExecutor & $179.71 ms$ & $506.8 ms^2$ & $22.51 ms$  \\
                    \hline
                    SmartExecutor & $10.31  ms$ & $9.04 ms^2$ & $3.01 ms$  \\
                    \hline
                    TunableExecutor & $10.08 ms$ & $7.91 ms^2$ & $2.81 ms$  \\
                    \hline                    
                \end{tabularx}  
        \end{table}
\end{frame}

\begin{frame}\frametitle{Результаты тестирования атомарного добавления 1}
    \begin{figure}[H]
        \centering
        \includegraphics[height=0.6\textheight]{../pics/simpleTran.png}
        \caption{График зависимости времени выполнения операций в ходе сеанса массовой загрузки(первый вариант)}
        \label{simpleTran}
    \end{figure}
\end{frame}

\begin{frame}\frametitle{Результаты тестирования атомарного добавления 2}
    \begin{figure}[H]
        \centering
        \includegraphics[height=0.6\textheight]{../pics/newTran.png}
        \caption{График зависимости времени выполнения операций в ходе сеанса массовой загрузки(второй вариант)}
        \label{newTran}
    \end{figure}
\end{frame}

\begin{frame} 
    \begin{center} 
        \Huge Спасибо за внимание! \\ Вопросы? 
    \end{center} 
    \begin{block}{Github repo:}
        https://github.com/cscenter/Vindur
    \end{block}
\end{frame} 

\end{document}